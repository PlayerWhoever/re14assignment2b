\documentclass{article}
\usepackage[utf8]{inputenc}

\title{Data Science}
\author{WEI Helong }
\date{June 2021}

\begin{document}


\maketitle

\section{Introduction}
I am a student majoring in data science. My elective course field is music. At the same time, the two elective courses I chose are CMPN1013 Creative Music Technology 3 and CMPN1014 Sound Recording Fundamentals. I will elaborate on them below.
\section{Why this elective area}
First of all, there are three reasons why I choose music as an elective field of data science: First, music has a lot of data, such as chords, musical styles, and popular tastes. We can collect data from major music platforms. Compared with other fields, its data has the advantages of easy collection, huge database, and fast enough data update. Second, the data of music has a clear direction, it can represent the aesthetic habits of the current public, and data science for music can bring the role of a weather vane to the entire industry, which is very meaningful. Third, the combination of music and data science is actually a fringe industry. Just as many people in this project will not choose the combination of music and data science, the combination of music and data science will not be the first choice of most people in real life. For this reason, such an industry has low competition, high demand, and high returns.
\section{What can we learn from these electives}
Then I want to talk about why I chose CMPN1013 Creative Music Technology 3 and CMPN1014 Sound Recording Fundamentals as two elective courses. The combination of music and data science can help our two major industries: music composition and audio synthesis. We use data to investigate the music tastes of the public, and study the writing of different styles of chords, and finally compose and use audio synthesis to complete a song. Therefore, the knowledge and ability we need for these two aspects is the knowledge of music creation and the basic ability of recording, and these two elective courses can help us very well.
\section{How these electives help us in major}
So how will these two elective courses help our major in data science? I think there are two aspects: First, we often need to understand the market trend before creating music, and we need to process the data provided by the music platform. Here we can use the knowledge of data science to help us determine the style of the song. Second, when we create songs, we need inspiration. The automatic music generation system developed by the expertise of musicians and computer experts can be directly written in computer primary and high-level programming languages, and through special digital audio and MIDI The communication interface is used for logical, random and artificial intelligence control of the connected digital sound source system and electronic musical instruments. Choose a predetermined mathematical model, and determine some variable parameters to deform, change, reproduce, and automatically generate works similar to a certain musician's style or a certain region, ethnic style, or works that are simply unimaginable. The source program written by it is equivalent to the score of a traditional composer, and the automatic generation is equivalent to the performance of a performer. The composer uses an expert system or an intelligent arrangement system to give the theme or the main melody, let the computer perform the orchestration of the basic part of the band, quickly complete the automatic generation, mode linking and triggering the sound source and other processes, and finally complete the production of the entire music to get twice the result with half the effort. 
\section{How these electives help us in profession}
As I said above, when we have the ability to combine music creation and data science, the knowledge and capabilities we acquire can create business value for us. For example, we can predict the trend of music trends and provide this information to some platforms. Such as spotify, tiktok, or help yourself to compose. In addition, we can make our own intelligent composition system and put this kind of artificial intelligence composition system into commercial use. Besides, through the processing and application of these music data, we have exercised our mastery and application ability of data science knowledge, and accumulated experience for us. Even if we did not engage in music-related work in the end, we also had many important experiences. Finally, these supplementary knowledge and abilities allow us to have many choices in the career direction, and we can choose to become a big data engineer. You can also choose to conduct research on sound signal processing, such as sound, speech, and brain waves. The research direction of sound is the separation and synthesis of music signals. The research direction of speech is speech signal recognition, subjective and objective evaluation. The research direction of brain waves is to convert brain wave signals with voice signals through wires, mainly for the disabled. We even become professional composers. Of course, only relying on these two elective courses is far from enough. In order to achieve a professional level, we still need to learn more knowledge of music theory and sound processing, but these two elective courses are the most representative.
\section{Any chance}
Finally, I want to talk about whether the elective courses can provide us with corresponding opportunities. As I said earlier, the combination of music and data science is actually a fringe industry, because this is a qualitative and quantitative study of art, a direction that is more difficult and has fewer employees. But because of this, it has the advantages of low competition, high demand, and high returns. As far as I know, the functions of music software, such as listening to songs and recognizing songs, and recommending songs through the songs you have listened to recently, are actually good examples of the combination of data science and music. You can classify music based on music theory knowledge and use computer skills to match the user’s playlist with the songs in the music library, so as to be close to the user’s preferences. Or you want to write an intelligent composition system by yourself with knowledge of basic code, machine learning, pattern recognition, data mining, etc. In fact, we can make these by ourselves, and we can even find some professional music students and even teachers to cooperate. Rather than expect others to provide opportunities, it is better to seize for yourself.
\end{document}