\documentclass{article}
\usepackage{geometry}
\usepackage{graphics}
\usepackage{apacite}
\usepackage{subfiles}
\usepackage{csquotes}
\geometry{legalpaper, margin=1in}


\begin{document}
\begin{center}
	\begin{huge}
		\textbf{INFO1111: Computing 1A Professionalism}\\
	\end{huge}  
	\vspace{10mm}
	\begin{huge}  
		\textbf{Semester 1 2021}\\  
	\end{huge}
	\vspace{10mm}
	\begin{huge}  
		\textbf{Project 2B}\\  
	\end{huge}
	\vspace{60mm}
	\begin{LARGE}  
		\textbf{Group Number: 3}\\  
	\end{LARGE}
	\vspace{10mm}
	\resizebox{100mm}{20mm}{
		\begin{tabular}{ |c|c|c| } 
			\hline
			 & Full Name & Student ID \\ 
			\hline
			1 & Qiyuan Sun & 500709005 \\ 
			\hline
			2 & Yuren Long & 500179312 \\ 
			\hline
			3 & Helong Wei & 510155494 \\ 
			\hline
			4 & Taoyi Xu & 510030629 \\ 
			\hline
		\end{tabular}
	} 
\end{center}
\pagebreak

\section{Introduction}
	\begin{displayquote}
	“Computer Science is no more about computers than astronomy is about telescopes.”— (Edsger W. Dijkstra, 1993)
	\end{displayquote}
	It is important for every student studying anything to not only build depth of knowledge with academic achievements, but also to widen scope into other areas that compliments their main direction of study. Whether it is through the 	lens of applications and implementations of their theories in other areas, through gaining professional knowledge that creates a comparative advantage relative to other graduates, or even through contrasting and reflecting the style of thinking and improve personally, studying in other electives with area outside of the focused major can be very beneficial.\par
	This report is made by four students in university of Sydney each assigned a computing major and tasked to recommend two electives from outside of computing. The contents contains:
	\begin{enumerate}
		\item a table of each team member’s assigned major
		\item each recommendations with details about area these electives are from, what the electives are, how it relates to the computing major, how it might be beneficial professionally, and the opportunities resulted from it. Here is a brief overlook of the majors and electives:
		\begin{itemize}
			\item Major of Information System complemented by BUSS1000: future of business and BUSS1020: quantitative business analysis.
			\item Major of Computer Science complemented by BADP1001: Empirical Thinking and BADP2001: Algorithmic Architecture.
			\item Major of Data Science complemented by CMPN1013 Creative Music Technology 3 and CMPN1014 Sound Recording Fundamentals.
		\end{itemize}
		\item The contributions for this project including the experiences of cooperation, reflection of the work, and subsequent improvements of collaboration skills.
	\end{enumerate}
	As mentioned before, the purpose the this report is to recommend different electives to a computing majoring student in order to assist in the professional and academic careers. The outcome, in addition to the purpose, has also improve each contributors’ git and latex skills, cooperation skills, professional representation skills, research skills, and  advanced comprehension beyond computing. To that end, each team member found satisfaction, gained knowledge, and developed capabilities to perform in the future.

	\pagebreak
\section{Major Allocation}
	\begin{center}
	\resizebox{120mm}{20mm}{
		\begin{tabular}{ |c|c|c| } 
			\hline
			 & Name & Major \\ 
			\hline
			1 & Qiyuan Sun & Information Systems \\ 
			\hline
			2 & Yuren Long & Software Development \\ 
			\hline
			3 & Helong Wei &  Data Science \\ 
			\hline
			4 & Taoyi Xu & Computer Science \\ 
		\hline
		\end{tabular}
	}
	\end{center}
\pagebreak

\section{Recommendations}
	\subsection{Computer Science}
		\subfile{CS.tex}
		\pagebreak
	\subsection{Data Science}
		\subfile{DS.tex}
		\pagebreak
	\subsection{Information Systems}
		\subfile{IS.tex}
		\pagebreak
	\subsection{Software Development}
		\subfile{SD.tex}
		\pagebreak

\section{Contributions}
	
	\pagebreak

\section{Bibliography}
	\nocite{*}
	\bibliographystyle{apacite}
	\bibliography{draftbib.bib}

\end{document}
