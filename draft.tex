\documentclass{article}
\usepackage{geometry}
\usepackage{graphics}
\usepackage{apacite}
\usepackage{subfiles}
\usepackage{csquotes}
\geometry{legalpaper, margin=1in}


\begin{document}
\begin{center}
	\begin{huge}
		\textbf{INFO1111: Computing 1A Professionalism}\\
	\end{huge}  
	\vspace{10mm}
	\begin{huge}  
		\textbf{Semester 1 2021}\\  
	\end{huge}
	\vspace{10mm}
	\begin{huge}  
		\textbf{Project 2B}\\  
	\end{huge}
	\vspace{60mm}
	\begin{LARGE}  
		\textbf{Group Number: 3}\\  
	\end{LARGE}
	\vspace{10mm}
	\resizebox{100mm}{20mm}{
		\begin{tabular}{ |c|c|c| } 
			\hline
			 & Full Name & Student ID \\ 
			\hline
			1 & Qiyuan Sun & 500709005 \\ 
			\hline
			2 & Yuren Long & 500179312 \\ 
			\hline
			3 & Helong Wei & 510155494 \\ 
			\hline
			4 & Taoyi Xu & 510030629 \\ 
			\hline
		\end{tabular}
	} 
\end{center}
\pagebreak
\section{Introdcution}
	\begin{displayquote}
	“Computer Science is no more about computers than astronomy is about telescopes.”— (Edsger W. Dijkstra, 1993)
	\end{displayquote}
	It is important for every student studying anything to not only build depth of knowledge with academic achievements, but also to widen scope into other areas that compliments their main direction of study. Whether it is through the 	lens of applications and implementations of their theories in other areas, through gaining professional knowledge that creates a comparative advantage relative to other graduates, or even through contrasting and reflecting the style of thinking and improve personally, studying in other electives with area outside of the focused major can be very beneficial.\par
	This report is made by four students in university of Sydney each assigned a computing major and tasked to recommend two electives from outside of computing. The contents contains:
	\begin{enumerate}
		\item a table of each team member’s assigned major
		\item each recommendations with details about area these electives are from, what the electives are, how it relates to the computing major, how it might be beneficial professionally, and the opportunities resulted from it. Here is a brief overlook of the majors and electives:
		\begin{itemize}
			\item Major of Information System complemented by BUSS1000: future of business and BUSS1020: quantitative business analysis.
			\item Major of Computer Science complemented by BADP1001: Empirical Thinking and BADP2001: Algorithmic Architecture.
			\item Major of Data Science complemented by CMPN1013: Creative Music Technology 3 and CMPN1014: Sound Recording Fundamentals.
			\item Major of Software Development complemented by ENGG111: Integrated Engineering 1 and DATA1002: Informatics: Data and Computation.
		\end{itemize}
		\item The contributions for this project including the experiences of cooperation, reflection of the work, and subsequent improvements of collaboration skills.
	\end{enumerate}
	This report is done using git (https://github.com/PlayerWhoever/re14assignment2b.git) to collabrate our work that is constructed through latex. \par
	As mentioned before, the purpose the this report is to recommend different electives to a computing majoring student in order to assist in the professional and academic careers. The outcome, in addition to the purpose, has also improve each contributors’ git and latex skills, cooperation skills, professional representation skills, research skills, and  advanced comprehension beyond computing. To that end, each team member found satisfaction, gained knowledge, and developed capabilities to perform in the future.
	\pagebreak

\section{Major Allocation}
	\begin{center}
	\resizebox{120mm}{20mm}{
		\begin{tabular}{ |c|c|c| } 
			\hline
			 & Name & Major \\ 
			\hline
			1 & Qiyuan Sun & Information Systems \\ 
			\hline
			2 & Yuren Long & Software Development \\ 
			\hline
			3 & Helong Wei &  Data Science \\ 
			\hline
			4 & Taoyi Xu & Computer Science \\ 
		\hline
		\end{tabular}
	}
	\end{center}
\pagebreak

\section{Recommendations}
	\subsection{Computer Science}
		\subfile{CS.tex}
		\pagebreak
	\subsection{Data Science}
		\subfile{DS.tex}
		\pagebreak
	\subsection{Information Systems}
		\subfile{IS.tex}
		\pagebreak
	\subsection{Software Development}
		\subfile{SD.tex}
		\pagebreak

\section{Contributions}
	\subsubsection{Management:}
	The team consider the management of the group to be relatively simple thanks to advanced technologies and lack of language barriers. Firstly, our communication is done through wechat, a multi-purpose messaging and social media app. Secondly, We implemented a milestone system where work is divided up into smaller fractions and each to be due on dedicated dates. lastly, the team leader would occasionally starts a zoom session to explain more complicated instructions and exemplify expected output of each member.
	\subsubsection{Teamwork:}
	\begin{enumerate}
		\item During planning, the team join a voice chat for to research for each major and electives. We discussed not only what electives suits the best and each shared their opinion but also provide thought provocative ideas and research for each other.
		\item During development stage, we use git as a collaborate version control system both to combine our individual works into a main file to compile into ultimate results and also to prevent deadly mistakes using branches to develop individually. 
		\item There is also use of zoom to demonstrate how git and latex is used properly to help others get started.
	During review stage, every member contributes critical feedbacks about the outcome of the report and each toke action to perfect their individual work.
	\end{enumerate}
	
	\subsubsection{Difficulties and challenges:}
	\begin{itemize}
		\item Not everyone is one the same level when using git and latex, branches operations and sub filing is confusing to some teammates.
		\item Helong Wei reported difficulties with accesses using git as he was originally using the business version of git.
		\item Taoyi Xu reported difficulties during formatting of bibtex which is lead by most of his sources is type online websites as it is difficult to access relevent scholar articles.
		\item Different style of writing that makes the report loose some cohesion, getting everyone to standardize on one style is challenging and ineffective.
		\item Bibtex merging resulted in several conflicts.
		\item Some areas and electives are not easy to find articles support the claims.
		\item Language barrier does take a role in both writing and researching as all four of the team members are Chinese, which cultural differences also has an affect on definition of “beneficial.”
		\item One incident of someone pushing not debugged program to remote master branch caused chaos.
	\end{itemize}
	
	\subsubsection{Things to change:}
	\begin{itemize}
		\item Branch and test before pushing into remote repository should be more strictly enforced upon.
		\item To have more scholar level article to support some of the claims is more beneficial.
		\item Writing in more presentable style (bullet points and line breaks) instead of raw paragraphs
		\item Have numeric statistics as evidence.
	\end{itemize}
	
	\subsubsection{Work distribution:}
	Each team member is meant to research and write a recommendation of electives outside of computing to students with the computing major assigned to each member. These acts as the body of the report while the team leader build and structured the report by building latex files that allows everyone to only need to work on their subfiles, settled up remote git repositories, and made sure each member have clear understanding of their work and have reasonable electives with solid research.
	
	\subsubsection{Contributions of each team members:}
	Qiyuan Sun is the team leader and is responsible for setting up the working environment such as recreating the report using the template given in pdf, development base files for each subfile and bibtex, creating and managing access to remote git repositories, and nevertheless writing introduction, contribution and the information system recommendations in the report.
	Helong wei is responsible for the data science recommendations in the report. He also contributed by providing thoughtful ideas during planning, and helping others understand the project better.
	Taoyi Xu is responsible for the Computer science recommendations in the report. He also contributed using quality researches and providing additional information in detail, he also provided many ideas for this contribution about our difficulties and what we should plan to improve on.
	Yuren Long is responsible for the software development recommendations in the report. He also contributed with reflective feedbacks on both planning stage and reviewing stage that further improved the quality of the work.
	\pagebreak
\section{Bibliography}
	\nocite{*}
	\bibliographystyle{apacite}
	\bibliography{draftbib.bib}

\end{document}
