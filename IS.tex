\documentclass[../draft.tex]{subfiles}
\begin{document}
	It is no denying that any academic subject cooperates two necessary stages, the theory and the application. This is especially true for the computing major of information system. I have chosen two electives that I would suggest a information system major student should study on: BUSS1000:future of business and BUSS1020: quantitative business analysis.
\subsubsection{area these electives are from}
And one of the best fit area for the application of information system is the area of business and commerce. As Walter Baets(1992) stated that 
\begin{displayquote}
"It is generally accepted that one of the key factors for successful IS planning and implementation is the close linkage of the IS strategy with business strategy."
\end{displayquote}
	As commerce can be defined as exchange of goods, services, or something of value, between entities. Which such entity can be recognized as a business which is responsible for the supply of such valuables and manage the exchanges. To which involves heavily around producing, exchanging, storing, representing and analyzing of the information.\par
\subsubsection{the electives are}
University of Sydney unit of study outlined the elective BUSS1000: Future of Business 
\begin{displayquote}
"This unit shows how to interpret data involving uncertainty and variability; how to model and analyse the relationships within business data; and how to make correct inferences from the data (and recognise incorrect inferences)."
\end{displayquote}
The elective discuss the general idea of not only how to predict the future of a business with analyzing pass information, but also focuses on how to plan the future for a scenario business and in turns helps students to development themselves just like the business entities they studied on.
University of Sydney unit of study outlined the elective BUSS1020: Quantitative business analysis as
\begin{displayquote}
"unit is designed to provide commencing undergraduate students with insights into the study and the practice of business."
\end{displayquote}
	That is to say it is a elective studying about how to manipulate quantities business information, store and apply them in a systematic way that is not only universally recognized but also efficient as many of it has been tested on uncountable real life cases.\par
\subsubsection{knowledge and capabilities will be developed by undertaking these electives}
As university of Sydney unit of study outline stated many learning outcomes:
\begin{displayquote}
"LO1. demonstrate an understanding of theory and conceptual frameworks that are relevant to businesses operating in diverse sectors
LO2. explain and critically assess the challenges and opportunities facing businesses in a variety of key sectors within the global economy 
LO3. apply conceptual frameworks to different business problems in order to derive insights about business performance and opportunities for strategic change" and etc
\end{displayquote}
\begin{itemize}
\item That is to say, BUSS1000 offers skills such as frameworks to apply information and analysis it with efficiency; 
\item planning skills that is specially important where there is lots and complicated information; 
\item knowledge on the stages through product cycles and identify key stakeholders;
\item and capability to co-work within a cooperation with the business’s point of view.
\end{itemize}
``Also according to university of Sydney unit of study outline, Buss1020 : Quantitative Business Analysis provide students with the ability that ”is important in all business disciplines since all disciplines deal with increasing amounts of data, and there are increasing expectations of quantitative skills.” Which in turns will builds a student’s foundation for processing large amount of data. 
\begin{itemize}
\item That is to say the course can develop the student with a sense of standards in collection of data so that the information can be recognized and stored not just easily by the collector but also his coworkers; 
\item a sense of representing complex and quantitative information efficiently that is able to not only be easily analysis individually, but also be easily understood by the consumer of the information(such as investors with financial reports); 
\item and a sense to develop quantitative analysis models that can be beneficial into other models where students can build a custom information system off of.
\end{itemize}
\subsubsection{how these electives might complement the knowledge and capabilities developed as part of the relevant major}
In the paper A business and ICT architecture for a logistics city, K.T.K.Toh, P.Nagel, and R.Oakden(2009) 
\begin{displayquote}
"has outlined the national logistics city project and presented the logistics information ecosystem solution that is just one of the enablers of the concept. The logistics EA is a key component of this solution that bridges the business architecture and the IT architecture."
\end{displayquote}
	Learning about business is very useful in vases variety of ways, since most students will eventually use their intellectual knowledge and capabilities developed through out their life to be earning money. Therefore having a whole picture of how their work would produce gross value is not only important as a direction during planning of the work, but also as a safety assurance to make sure that one can understand their work’s true value.\par
To that end, business and commerce is especially important for a information system majoring students. Firstly, they need to understand the type and the purpose of the information, where business and commerce excels at in the application stage. Secondly, they need to identify how to process the information and fit then into the right model, which buss1000 emphasize. Thirdly, the model must be developed in the most efficient way which is covered in bus1020. Fourthly the ability to develop a system out of the information given can also be stimulated as both the electives builds on a structural thinking. And lastly, both of the electives have strong emphasize on analysis which is the ultimate process for a information, this also work excellent in reviewing the system.\par
\subsubsection{if the knowledge and capabilities developed by undertaking these electives will assist in your professional career}
According to A. Levas; P. Jain; S. Boyd; W.A. Tulskie (1995) 
\begin{displayquote}
"Modeling and simulation are important technologies that can be applied to business process reengineering (BPR). Dynamic process models afford the analysis of alternative process scenarios through simulation by providing quantitative process metrics"
\end{displayquote}
	Studying in BUSS1000 would help tremendous with a professional career not only in IT, but also any firms that requires an IT person. As the student not only improve their range of skill sets which mostly focused on application of the theories, an information system major graduate can also find more deeps in their professional skills as there is more dimension to their scope for the information, and a clearer picture to identify and categorize information in the professional skim, which in turns improves their architectural skills when building a system of information.\par
Buss1020 will give one graduate more practice of how their theories can be applied before going into their professional careers, as many of information system graduates might be allocated with systematically storing cooperate information and consumer feedbacks which will be processed and analysis latter. Having experience on working with business models and incorporating it with advanced computing is very advantageous.\par
\subsubsection{if there are any opportunities resulting from undertaking these electives}
It has been common cooperate practice, since almost 30 years ago, that raw business quantitative information should be processed by computers into useful information. Therefore it is common that one business should have a information technology staff and that staff is very advantageous if one have a background or at least some experience business. So one can actually understand the purpose and application of the graph one produced, and thus perform better. 
This also works in a technology focused firm, the information system majored employee would have an idea of what kind of information he should communicate with the financial department.\par
The biggest opportunity of understanding business besides majoring in information technology is, however, to start your own business. IT is one of the best industries to start off a business with nothing but an innovating app or algorithm, due to it having very low entry cost to have your program published and start making profits. Therefore there will be a relatively higher possibility that one information system majored graduate would choose to be a entrepreneur, and having basic knowledge of business through BUSS1000 and having capability on application from raw prototype of a program into profitable commercial applications through BUSS1020 is a massive comparative advantage as one might just beat fellow graduates with the same idea to market.\par
\subsubsection{In conclusion}
I consider that studying in BUSS1000: future of business and BUSS1020: quantitative business analysis is best suited for a information system student as the two electives gives promising opportunities, helps to build wider scope in turns of the range of one’s skill sets, and further develops on depths in professional skills as the business style of thinking is very much reflecting where one can explore their weakness better and make use of their comparative advantages not just in their professional career, but also in their academic advancements.\par
\end{document}
